%!TeX root=../tese.tex
%("dica" para o editor de texto: este arquivo é parte de um documento maior)
% para saber mais: https://tex.stackexchange.com/q/78101/183146

% Insira aqui os metadados do seu trabalho. Para isso, copie,
% com as alterações necessárias, o conteúdo do arquivo
% conteudo-exemplo/metadados.tex

%%%%%%%%%%%%%%%%%%%%%%%%%%%%%%%%%%%%%%%%%%%%%%%%%%%%%%%%%%%%%%%%%%%%%%%%%%%%%%%%
%%%%%%%%%%%%%%%%%%%%%%%%%%%%% METADADOS DA TESE %%%%%%%%%%%%%%%%%%%%%%%%%%%%%%%%
%%%%%%%%%%%%%%%%%%%%%%%%%%%%%%%%%%%%%%%%%%%%%%%%%%%%%%%%%%%%%%%%%%%%%%%%%%%%%%%%

% Estes comandos definem o título e autoria do trabalho e devem sempre ser
% definidos, pois além de serem utilizados para criar a capa, também são
% armazenados nos metadados do PDF.
\title{
    % Obrigatório nas duas línguas
    titlept={O Processo de Desenvolvimento de uma CPU RISC-V},
    titleen={The Process of Developing a RISC-V CPU},
    % Opcional, mas se houver deve existir nas duas línguas
    %subtitlept={um subtítulo},
    %subtitleen={a subtitle},
    % Opcional, para o cabeçalho das páginas
    shorttitle={Densenvolvendo uma CPU RISC-V},
}

\author[fem]{Elisa Uhura Pereira da Silva}

% Para TCCs, este comando define o supervisor
\orientador[masc]{Prof. Dr. Alfredo Goldman}

% Se não houver, remova; se houver mais de um, basta
% repetir o comando quantas vezes forem necessárias
%\coorientador{Prof. Dr. Ciclano de Tal}
%\coorientador[fem]{Profª. Drª. Beltrana de Tal}

% A página de rosto da versão para depósito (ou seja, a versão final
% antes da defesa) deve ser diferente da página de rosto da versão
% definitiva (ou seja, a versão final após a incorporação das sugestões
% da banca).
\defesa{
  nivel=tcc, % mestrado, doutorado ou tcc
  % É a versão para defesa ou a versão definitiva?
  %definitiva,
  % É qualificação?
  %quali,
  programa={Ciência da Computação},
  membrobanca={Profª. Drª. Fulana de Tal (orientadora) -- IME-USP [sem ponto final]},
  % Em inglês, não há o "ª"
  %membrobanca{Prof. Dr. Fulana de Tal (advisor) -- IME-USP [sem ponto final]},
  membrobanca={Prof. Dr. Ciclano de Tal -- IME-USP [sem ponto final]},
  membrobanca={Profª. Drª. Convidada de Tal -- IMPA [sem ponto final]},
  % Se não houver, remova
  %apoio={Durante o desenvolvimento deste trabalho o autor
  %       recebeu auxílio financeiro da XXXX},
  local={São Paulo},
  data=2021-12-22, % YYYY-MM-DD
  % A licença do seu trabalho. Use CC-BY, CC-BY-NC, CC-BY-ND, CC-BY-SA,
  % CC-BY-NC-SA ou CC-BY-NC-ND para escolher a licença Creative Commons
  % correspondente (o sistema insere automaticamente o texto da licença).
  % Se quiser estabelecer regras diferentes para o uso de seu trabalho,
  % converse com seu orientador e coloque o texto da licença aqui, mas
  % observe que apenas TCCs sob alguma licença Creative Commons serão
  % acrescentados ao BDTA.
  direitos={CC-BY}, % Creative Commons Attribution 4.0 International License
  %direitos={Autorizo a reprodução e divulgação total ou parcial
  %          deste trabalho, por qualquer meio convencional ou
  %          eletrônico, para fins de estudo e pesquisa, desde que
  %          citada a fonte.},
  % Isto deve ser preparado em conjunto com o bibliotecário
  %fichacatalografica={nome do autor, título, etc.},
}

% As palavras-chave são obrigatórias, em português e
% em inglês. Acrescente quantas forem necessárias.
\palavrachave{RISC-V}
\palavrachave{Verilog}
\palavrachave{Objective-C}
\palavrachave{FPGA}

\keyword{RISC-V}
\keyword{Verilog}
\keyword{Objective-C}
\keyword{FPGA}

% O resumo é obrigatório, em português e inglês.
\resumo{
A RISC-V é uma arquitetura de conjunto de instruções
aberta e disponível de forma gratuita para uso na indústria 
e na academia. Com o objetivo de ser a arquitetura padrão 
para todos os dispositivos computacionais, sua diferença
principal é a disponibilidade de diversas implementações abertas desenhadas
para uso em diversos tipos de tarefas computacionais.
Este trabalho apresenta o processo de desenvolvimento 
de um processador usando a arquitetura RISC-V com 
enfoque na parte de verificação do circuito. O processo foi divido
nas etapas de \emph{planejamento e preparação}, 
\emph{desenvolvimento da descrição de circuito} e \emph{verificação}, e
o seu sistema de testes faz uso da linguagem \emph{Objetive-C},
permitindo o uso de técnicas escritas de teste de unidade 
combinadas com técnicas de testes de circuitos digitais.
}

\abstract{
  RISC-V is an open instruction set architecture
  available for free for both industry and academic use.
  Aiming to be the default architecture
  for all computing devices, it differs from other architectures
  by the availability of several open implementations
  designed for use in many kinds of computing tasks.
  This work presents the development process
  of a RISC-V processor focusing on the design verification phase. 
  The process has been split into the stages of \emph{planning and preparation},
  \emph{circuit description development}, and
  \emph{verification}. The test system uses 
  the Objective-C language, allowing the use of unit test techniques
  combined with digital circuit testing techniques.
}
