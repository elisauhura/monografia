%!TeX root=../tese.tex
%("dica" para o editor de texto: este arquivo é parte de um documento maior)
% para saber mais: https://tex.stackexchange.com/q/78101/183146

%% ------------------------------------------------------------------------- %%
\chapter{Introdução}
\label{cap:introducao}

A RISC-V é uma arquitetura de conjunto de instruções (ISA do inglês 
\emph{instruction set architecture}) aberta e disponível de forma gratuita 
tanto para uso na indústria quanto na academia. 
Ela é padronizada e patrocinada pela \emph{RISC-V International}, uma organização 
internacional sem fins lucrativos que possui dentre seus membros grandes 
empresas do setor de tecnologia como \emph{Google}, \emph{Alibaba Cloud} e 
\emph{Western Digital} \citep{MEM}.

Ela teve seu início em 2010 através do trabalho do professor Krste Asanović e os 
alunos de pós graduação Yunsup Lee e Andrew Waterman no \emph{Parallel Computing
Laboratory} na \emph{UC Berkeley} \citep{RVH}, sendo desde o começo disponível 
através de alguma licença aberta.

O \emph{V} no nome do padrão se refere ao número 5, fazendo com que as pronúncias
\emph{risc-cinco} e \emph{risc-five} sejam mais adequadas que \emph{risqc-ve}.
O número refere-se ao fato da RISC-V ser a quinta grande ISA desenvolvida na 
\emph{UC Berkeley}, referência importante pois tendo como objetivo ser a ISA 
padrão para todos os dispositivos computacionais \citep{Asanović:EECS-2014-146}, 
diversas decisões de design da arquitetura foram tomadas considerando decisões 
realizadas por arquiteturas anteriores e suas implicações.

Um exemplo é a ausência de instruções com \emph{delay slot}, uma técnica usada 
em arquiteturas como a MIPS \citep{DSTL}, onde instruções logo após uma instrução
de pulo são executadas antes do pulo ser executado. Esse tipo de decisão de design
leva ao favorecimento de implementações que fazem uso de técnicas de \emph{pipeline},
aumentando a complexidade de implementações \emph{in-order} da arquitetura.

A arquitetura conta com o suporte a diversas linguagens de programação como \emph{Go}
\citep{GOL}, \emph{C/C++} \citep{RVGCC} e linguagens que fazem uso da \emph{LLVM} 
\citep{RVLLVM}, bem como o suporte de sistemas operacionais como o Debian \citep{RVDB}. 
Um diferencial da arquitetura é a disponibilidade de núcleos de propriedade intelectual
de semicondutores (IP do inglês \emph{intellectual property}) abertos para os mais 
diversos casos de uso, como o \emph{PicoRV} \citep{PICO} para sistemas embarcados, 
o \emph{Vortex} \citep{elsabbagh2020vortex} que usa núcleos RISC-V como aceleradores 
gráficos ou o \emph{Xuantie-910} \citep{9138983} destinado ao uso em servidores.

O desenvolvimento de IPs é feito usando uma linguagem de descrição de hardware 
(HDL do inglês \emph{hardware description language}) sendo a \emph{Verilog} e a 
\emph{VHDL} as mais bem suportadas na indústria. Com base na descrição é possível 
simular o comportamento do IP ou sintetizar um configuração para uso em FPGAs ou 
para fabricação de um \emph{chip}.

Matriz de porta programáveis (FPGA do inglês \emph{field-programmable gate array}) é um 
dispositivo constituído principalmente por blocos de lógica programável (CLB do inglês 
\emph{configurable logic block}), interconectores e blocos de entrada e saída 
(IOB do inglês \emph{input output block}) que são configurados para implementar um dado
circuito lógico.

A primeira FPGA, a \emph{XC2064}, foi introduzida em 1984 pela \emph{Xilinx} \citep{8392473}
contendo 64 CBLs \emph{XCSP}. A \emph{7 series}, uma família de FPGAs lançada em 2010 pela \emph{Xilinx} 
\citep{S7LN} conta com modelos com até 305 mil CBLs \citep{7CLB}, sendo as CLBs da \emph{7 series}
capazes de representar circuitos mais complexos que as CLBs da \emph{XC2064}.

O alto custo inicial para a fabricação de um \emph{chip} em relação a aquisição de FPGAs e o aumento dos
circuitos que podem ser implementados nelas permitem que elas sejam usadas tanto no processo de 
desenvolvimento de um circuito integrado (CI) quanto no produto final dependendo da escala do projeto
através da adição de alguma memória que guarda a configuração desejada para ser carregada ao ligar a FPGA.

O desenvolvimento de um processador envolve diversas etapas desde sua concepção até a sintetização para
uso em uma FPGA ou produção de um \emph{chip}. Além da escolha da ISA, é necessário definir detalhes do ambiente 
de execução que não são definidos pela arquitetura, escolha do design do processador, desenvolvimento da 
descrição do hardware, uso de testes para verificar a descrição e análises para garantir o comportamento
correto do circuito e outras atividades relacionadas a fabricação do \emph{chip} em si.

O objetivo deste trabalho é registrar o processo de desenvolvimento de um processador usando a arquitetura 
RISC-V desde a definição do ambiente até a síntese para uso em uma FPGA com um enfoque na etapa de verificação 
através do testes com base na simulação dos componentes, um etapa que não tende a ser muito explorada em materiais 
introdutórios sobre o desenvolvimento de processadores.

O capítulo 2 aborda detalhes da notação e conceitos utilizados ao longo do texto. Seguido pelo capítulo 3 que apresenta uma descrição
da parte de interesse da arquitetura. O captítulo 4 datalha as linguagens \emph{Verilog} e \emph{Objective-C}
e a FPGA \emph{XC7A100T} utilizada no projeto. O capítulo 5 descreve o desenvolvimento e estruturação do processador.
O capítulo 6 apresenta resultados da análise de desempenho. Por fim o capítulo 7 conclui os resultados e apresenta
algumas reflexões sobre o desenvolvimento da CPU.
