%!TeX root=../tese.tex
%("dica" para o editor de texto: este arquivo é parte de um documento maior)
% para saber mais: https://tex.stackexchange.com/q/78101/183146

%% ------------------------------------------------------------------------- %%
\chapter{Introdução}
\label{cap:introducao}

\section{Contextualização}
\label{sec:ic}

A RISC-V é uma arquitetura de conjunto de instruções (ISA do inglês 
\emph{instruction set architecture}) aberta e disponível de forma gratuita 
tanto para uso na indústria quanto na academia. 
Ela é padronizada e patrocinada pela \emph{RISC-V International}, uma organização 
internacional sem fins lucrativos que possui dentre seus membros grandes 
empresas do setor de tecnologia como \emph{Google}, \emph{Alibaba Cloud} e 
\emph{Western Digital} \citep{MEM}.

Ela teve seu início em 2010 através do trabalho do professor Krste Asanović e os 
alunos de pós graduação Yunsup Lee e Andrew Waterman no \emph{Parallel Computing
Laboratory} na \emph{UC Berkeley} \citep{RVH}, sendo desde o começo disponível 
através de alguma licença aberta. Com o tempo o projeto evoluiu e a ISA passou 
a incorporar o objetivo de ser a arquitetura padrão para todos os dispositivos 
computacionais \citep{Asanović:EECS-2014-146}.

Assim as decisões de design da arquitetura são tomadas considerando decisões 
realizadas por arquiteturas anteriores e suas implicações. Desse modo, a RISC-V
evita \emph{erros} cometidos por arquiteturas anteriores.

Um exemplo é a ausência de instruções com \emph{delay slot}, uma técnica usada 
em arquiteturas como a MIPS \citep{DSTL}, onde instruções logo após uma instrução
de pulo são executadas antes do pulo ser executado. Esse tipo de decisão de design
favorece implementações que fazem uso de técnicas de \emph{pipeline}, que eram
populares na época, porém aumenta a complexidade de implementações \emph{in-order} 
da arquitetura.

O diferencial da arquitetura é a disponibilidade de núcleos de propriedade intelectual
de semicondutores (IP do inglês \emph{intellectual property}) abertos para os mais 
diversos casos de uso, como o \emph{PicoRV} \citep{PICO} para sistemas embarcados, 
o \emph{Vortex} \citep{elsabbagh2020vortex} que usa núcleos RISC-V como aceleradores 
gráficos ou o \emph{Xuantie-910} \citep{9138983} destinado ao uso em servidores.

Além da disponibilidade de IPs, a arquitetura conta com o suporte de diversas 
linguagens de programação como \emph{Go} \citep{GOL}, \emph{C/C++} \citep{RVGCC} e
linguagens que fazem uso da \emph{LLVM} \citep{RVLLVM}, bem como o suporte de sistemas 
operacionais como o Debian \citep{RVDB}.

\section{Objetivos}
\label{sec:io}

O objetivo deste trabalho é registrar o processo de desenvolvimento de um processador 
usando a arquitetura RISC-V desde a definição do ambiente até a síntese para uso em uma 
matriz de porta programáveis com um enfoque na etapa de verificação através do testes 
com base na simulação dos componentes.

\section{Metodologia}
\label{sec:im}

O desenvolvimento de IPs é feito usando uma linguagem de descrição de \emph{hardware} 
(HDL do inglês \emph{hardware description language}) sendo a \emph{Verilog} e a 
\emph{VHDL} as mais bem suportadas na indústria. Com base na descrição é possível 
simular o comportamento do IP ou sintetizar um configuração para uso em 
matriz de porta programáveis (FPGA do inglês \emph{field-programmable gate array}) ou 
para fabricação de um \emph{chip}.

O desenvolvimento de um processador envolve diversas etapas desde sua concepção até a sintetização para
uso em uma FPGA ou produção de um \emph{chip}. 
Além da escolha da ISA, é necessário definir detalhes do ambiente 
de execução que não são definidos pela arquitetura, escolha do design do processador, desenvolvimento da 
descrição do \emph{hardware}, uso de testes para verificar a descrição e análises para garantir o comportamento
correto do circuito.

Este trabalho foi divido nas etapas de \emph{planejamento e preparação}, 
\emph{desenvolvimento da descrição} e \emph{verificação}.
Na etapa de \emph{planejamento e preparação} foi definida as tecnologias utilizadas,
estruturação do projeto e desenvolvimento de bibliotecas auxiliares. 
Na etapa de \emph{desenvolvimento da descrição} foi desenvolvido os módulos que constituem o
processador. E na etapa
\emph{verificação} foram desenvolvidos testes para validar o comportamento do \emph{hardware}.

\section{Organização do texto}
\label{sec:it}

O texto desta monografia é dividido em capítulos que podem apresentações seções e subseções. 
O Capítulo~\ref{cap:2} aborda a notação para manipulação de vetores e conceitos relacionado,
a parte utilizada da arquitetura RISC-V, apresenta uma introdução às linguagens \emph{Verilog}
e \emph{Objective-C} e por fim descreve as partes constituintes de uma FPGA.

No Capítulo~\ref{cap:3} é descrita a estrutura e \emph{design} do processador e o sistema de testes.
O Capítulo~\ref{cap:4} conclui a monografia com resultados de análise de desempenho do processador 
e reflexões sobre o desenvolvimento da CPU.