%!TeX root=../tese.tex
%("dica" para o editor de texto: este arquivo é parte de um documento maior)
% para saber mais: https://tex.stackexchange.com/q/78101/183146

%% ------------------------------------------------------------------------- %%
\chapter{Projeto do processador e sistema de testes}
\label{cap:3}

\section{EEI}
\label{sec:eei}

Para o projeto foi escolhido desenvolver um processador com um \emph{hart} físico RV32ICZicsr e 
uma interface de comunicação serial para processamento de informações. O ambiente de execução
suporta acessos não alinhados na memória e disponibiliza os níveis de privilégio de máquina
usuário.

\subsection{Memória}
\label{ssec:msr}

O sistema é composto por um \emph{hart} com extremidade \emph{little-endian}. A memória possui 4
regiões identificadas pelos símbolos sA, sB, sC e sD. Considerando $mem^{8'2^\text{XLEN}}$, a Tabela~\ref{tab:memregions} 
descreve o intervalo
de memória de cada região e sua funcionalidade. Regiões da memória não coberta por essas regiões
sempre possuem o valor de 0 e o valor do $pc$ não pode apontar para elas.

\begin{table}
  \begin{tabular}{ |p{0.08\linewidth}|p{0.25\linewidth}|p{0.5\linewidth}| } 
    \hline
    Região & Intervalo & Descrição \\ \hline \hline
    sA & mem[h280:h7F] & Região apenas de leitura utilizada para armazenar a BIOS do sistema. 
    O $pc$ só pode apontar para essa região de memória caso ele esteja no nível de máquina. \\ \hline

    sB & mem[hF00:F0F] & Região utilizada para mapeamento de dispositivos de entrada e saída. O $pc$ não pode apontar para essa região da memória
    e ela só pode ser lida ou escrita pelo \emph{hart} caso ele esteja no nível de máquina. \\ \hline

    sC & mem[h2000:h20FFF] & Região de memória com possibilidade de leitura e escrita caso o \emph{hart} esteja no nível de memória.
    O $pc$ só pode apontar para essa região de memória caso ele esteja no nível de máquina. \\ \hline

    sD & mem[h8000:h8EFFF] & Região de memória com possibilidade de leitura e escrita independente do nível do  \emph{hart} esteja no nível de máquina.
    Destinada para armazenamento do programa que executa com privilégios. \\ \hline
  \end{tabular}
\caption{Tabela de regiões da memória disponibilizadas pelo \emph{EEI} \label{tab:memregions}}
\end{table}

A memória da região sB é utilizada para mapear a interface serial, o temporizador de máquina, o sistema de bancos de memória e leitura dos pinos de entrada e
saída.

\subsubsection{Interface Serial}
\label{ssec:serialio}

A interface serial é utilizada para comunicação assíncrona com dispositivos externos. Sua comunicação é feita através do envio e recebimento de \emph{bytes}
com uma velocidade de  de 9600 Baud utilizando o protocolo \emph{universal asynchronous receiver-transmitter} (UART) sem \emph{bits} de paridade e 1 
\emph{bit} de parada. A comunicação utiliza filas de 8 bytes para entrada e saída, permitindo que a comunicação seja processada em blocos de 8 bytes.
A interface é mapeada em sB[3:0] onde:
\begin{itemize}
  \item sB[0] vale 1 caso existam \emph{bytes} a serem lidos, 3 caso a fila de leitura esteja cheia e 0 caso não tenham bytes disponíveis para a leitura.
  \item sB[1] vale 1 caso existam \emph{bytes} a fila de envio tenha espaço, 3 caso a fila de vazio esteja vazia e 0 caso ela esteja cheia.
  \item sB[2] quando lido retorna o valor da cabeça da fila de leitura e remove o valor da fila.
  \item sB[3] quando escrito adiciona o valor na fila de envio.
\end{itemize}

A interface serial suporta apenas leituras e escrita através das instruções LB e SB.

\subsubsection{Temporizador}
\label{ssec:timerm}

O intervalor sb[11:4] é utilizado para armazenar o temporizador de máquina. Caso o valor dele seja menor ou igual ao do contador de máquina
é emitida uma interrupção de temporizador de máquina.

\subsubsection{Sistema de bancos}
\label{ssec:bancos}

A memória da região sD utiliza um sistema de bancos onde trechos de memória de 7800 \emph{bytes} constituem um banco de memória. sB[14] só pode ser lido 
e guarda a quantidade de bancos disponíveis e sempre é um valor maior ou igual a 2. 
sB[12] indica o banco que é mapeado em sD[77FF:0] e sB[13] indica o banco que é
mapeado em sD[EFFF:7800]. Os bancos são indexados a parir do zero, assim os valores de sB[12] e sB[13] devem ser sempre menor que sB[14] e a tentiva de
escrita de um valor inválido é ignorada. Além disso sB[12] e sB[13] podem apontar para o mesmo banco.

\subsubsection{Pinos de entrada e saída}
\label{ssec:gpio}

O byte sB[15] é utilizado para mapear pinos de entrada e saída, onde sB[15][7:4] é mapeado em pinos de saída e sB[15][3:0] é mapeado em pinos de entrada.
O funcionamento desses pinos é descrito na Seção~\ref{sec:sin}.

\subsection{Inicialização}
\label{ssec:init}

Ao iniciar o \emph{hart} pula para a posição h280 e começa a execução do programa da BIOS em modo de máquina com interrupções desativadas.
O valor do temporizador de máquina é alterado pra hFFFFFFFFFFFFFFFF e sB[12] e sB[13] valem 0.

\section{Design do Processador}
\label{sec:ddp}

Para o desenvolvimento do processador foi escolhido criar uma microarquitetura \emph{in-order muticycle} como a descrita por \cite{harris2021digital}.
O processador implementa uma máquina de estados 

\begin{figure}
  \centering
  \begin{tikzpicture}
    \node[state, initial] (q1) {Inicialização};
  \end{tikzpicture}
  \caption{Máquina de estados do processador}\label{fig:graficos}
\end{figure}

\subsection{Unidade de controle}
\label{ssec:udc}

\subsection{Unidade de decodificação}
\label{ssec:decode}

\subsection{Unidade de processamento aritimético}
\label{ssec:upa}

\subsection{Arquivo de registro}
\label{ssec:memfile}

\subsection{Interface de memória}
\label{ssec:memint}

\subsection{Interface serial}
\label{ssec:serial}

\section{Estrutura do Projeto}
\label{sec:edp}

\subsection{Módulos \emph{Verilog}}
\label{ssec:mverilog}

\subsection{Interface de módulos}
\label{ssec:imdl}

\subsection{Classes de apoio}
\label{ssec:apoio}

\section{Sistema de testes}
\label{sec:sdt}

\subsection{Representação de um módulo}
\label{sec:rdm}

\subsection{\emph{Script} de testes}
\label{ssec:script}

\subsection{\emph{Testbench}}
\label{ssec:bench}

\subsection{Execução de testes e integração contínua}
\label{sec:edt}

\section{Sintetização}
\label{sec:sin}


