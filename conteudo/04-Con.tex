%!TeX root=../tese.tex
%("dica" para o editor de texto: este arquivo é parte de um documento maior)
% para saber mais: https://tex.stackexchange.com/q/78101/183146

%% ------------------------------------------------------------------------- %%
\chapter{Conclusão}
\label{cap:4}

Desenvolver um processador RISC-V é uma tarefa extensa, que exige uma ampla gama
de conhecimento para ser realizada. A princípio, a ISA ser aberta se mostra como
fator convidativo, em contraste com conjuntos de instruções mais populares, que
costumam ser proprietários, como ARM e x86. No entanto, a RISC-V é flexível para
atender casos de uso bastante variados e para oferecer a tal característica, coloca no
desenvolvedor a responsabilidade de tomar diversas decisões acerca do ambiente de
execução, tais como: \emph{endianness}, organização da memória e mapeamento de dispositivos
de entrada e saída. A fim de agilizar o projeto, o desenvolvedor pode optar por utilizar
os padrões definidos em outras implementações ou até mesmo iniciar sua própria
implementação a partir de uma de código já existente.

Definido o conjunto de instruções oferecidos pelo processador, é preciso ponderar
sobre a maneira com que as instruções serão executadas, os requisitos do projeto devem
guiar essa escolha. Apesar de técnicas como \emph{pipelining} oferecerem
ganhos substanciais de desempenho, o projeto é penalizado com o aumento de complexidade, 
prolongando sua execução. Portanto, é uma boa prática ater-se à uma microarquitetura menos 
complexa, contanto que atenda aos requisitos impostos.

Com os detalhes do processador definidos, o projeto deve ser estruturado de modo adequado 
para o desenvolvimento e verificação dos módulos. Enquanto o uso de uma HDL costuma ser 
satisfatório para a descrição do \emph{hardware}, a verificação é compreendida por etapas em que
apenas o seu uso pode ser contra produtivo. Devido ao aumento exponencial do número de estados
em relação ao aumento do número de células de memória no projeto, o tempo gasto na verificação
acaba sendo muito maior que o tempo gasto na descrição dos componentes, e assim, a dedicação
de um tempo maior para o desenvolvimento de um sistema de testes mais sofisticado é
justificada com a redução no tempo para a escrita das verificações.

\section{Próximos passos}
\label{sec:aaaaaaaaa}

Em relação ao processador, é possível adicionar mais extensões,
como a extensão M, que acelera operações matemáticas com valores inteiros, e
expandir o modelo da memória para ter acesso a outros dispositivos de entrada e saída,
ou módulos de memória DDR3, aumentando o espaço disponível para programas.
No que se refere ao sistema de teste, é interessante explorar em iterações futuras
a implementação de roteiros mais complexos para simplificar a escrita
de testes que fazem uso de técnicas, como \emph{fuzzing}, utilizada para 
cobrir um número maior de estados durante os testes.
